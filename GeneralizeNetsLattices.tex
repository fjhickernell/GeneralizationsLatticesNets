\documentclass[12pt]{amsart}

\hoffset -0.7in
\textwidth 6.5in

%% Language and font encodings
\usepackage[english]{babel}
\usepackage[utf8x]{inputenc}
\usepackage[T1]{fontenc}


%% Useful packages
\usepackage{amsmath,booktabs}
\usepackage{graphicx}
\newtheorem{lemma}{Lemma}

\DeclareSymbolFont{GreekLetters}{OML}{cmr}{m}{it} %Provide missing letters
\DeclareSymbolFont{UpSfGreekLetters}{U}{cmss}{m}{n} %Provide missing letters
\DeclareMathSymbol{\varrho}{\mathalpha}{GreekLetters}{"25}
\DeclareMathSymbol{\UpSfLambda}{\mathalpha}{UpSfGreekLetters}{"03}
\DeclareMathSymbol{\UpSfSigma}{\mathalpha}{UpSfGreekLetters}{"06}
%\newcommand{\bvec}[1]{\boldsymbol{#1}}
\providecommand{\mathbold}{\boldsymbol}
\newcommand{\bvec}[1]{\mathbold{#1}}
%\newcommand{\bvec}[1]{\text{\boldmath$#1$}}
\newcommand{\avec}[1]{\vec{#1}}
%\renewcommand{\vec}[1] {\text{\boldmath$#1$}}
%\renewcommand{\vec}[1]{\ensuremath{\mathbf{#1}}}
%\newcommand{\vecsym}[1]{\ensuremath{\boldsymbol{#1}}}
\newcommand{\vecsym}[1]{\ensuremath{\mathbold{#1}}}
\def\bbl{\text{\boldmath$\{$}}
\def\bbr{\text{\boldmath$\}$}}
\newcommand{\bbrace}[1]{\bbl #1 \bbr}
\newcommand{\bbbrace}[1]{\mathopen{\pmb{\bigg\{}}#1\mathclose{\pmb{\bigg\}}}}
%\def\bbl{\boldsymbol{\left \{}}
%\def\bbr{\boldsymbol{\right \}}}
\def\betahat{\hat\beta}
%\def\e{\text{e}}
%\def\E{\text{E}}
\newcommand{\dif}{{\rm d}}

\newlength{\overwdth}
\def\overstrike#1{ 
\settowidth{\overwdth}{#1}\makebox[0pt][l]{\rule[0.5ex]{\overwdth}{0.1ex}}#1}

\def\abs#1{\ensuremath{\left \lvert #1 \right \rvert}}
\newcommand{\normabs}[1]{\ensuremath{\lvert #1 \rvert}}
\newcommand{\bigabs}[1]{\ensuremath{\bigl \lvert #1 \bigr \rvert}}
\newcommand{\Bigabs}[1]{\ensuremath{\Bigl \lvert #1 \Bigr \rvert}}
\newcommand{\biggabs}[1]{\ensuremath{\biggl \lvert #1 \biggr \rvert}}
\newcommand{\Biggabs}[1]{\ensuremath{\Biggl \lvert #1 \Biggr \rvert}}
\newcommand{\norm}[2][{}]{\ensuremath{\left \lVert #2 \right \rVert}_{#1}}
\newcommand{\normnorm}[2][{}]{\ensuremath{\lVert #2 \rVert}_{#1}}
\newcommand{\bignorm}[2][{}]{\ensuremath{\bigl \lVert #2 \bigr \rVert}_{#1}}
\newcommand{\Bignorm}[2][{}]{\ensuremath{\Bigl \lVert #2 \Bigr \rVert}_{#1}}
\newcommand{\biggnorm}[2][{}]{\ensuremath{\biggl \lVert #2 \biggr \rVert}_{#1}}
\newcommand{\ip}[3][{}]{\ensuremath{\left \langle #2, #3 \right \rangle_{#1}}}

\newcommand{\bigvecpar}[3]{\ensuremath{\bigl ( #1 \bigr )_{#2}^{#3}}}
\newcommand{\Bigvecpar}[3]{\ensuremath{\Bigl ( #1 \Bigr )_{#2}^{#3}}}
\newcommand{\biggvecpar}[3]{\ensuremath{\biggl ( #1 \biggr )_{#2}^{#3}}}
\newcommand{\bigpar}[1]{\ensuremath{\bigl ( #1 \bigr )}}
\newcommand{\Bigpar}[1]{\ensuremath{\Bigl ( #1 \Bigr )}}
\newcommand{\biggpar}[1]{\ensuremath{\biggl ( #1 \biggr )}}

\newcommand{\IIDsim}{\overset{\textup{IID}}{\sim}}

\DeclareMathOperator{\success}{succ}
\DeclareMathOperator{\sinc}{sinc}
\DeclareMathOperator{\sech}{sech}
\DeclareMathOperator{\csch}{csch}
\DeclareMathOperator{\dist}{dist}
\DeclareMathOperator{\spn}{span}
\DeclareMathOperator{\sgn}{sgn}
\DeclareMathOperator*{\rmse}{rmse}
\DeclareMathOperator{\Prob}{\mathbb{P}}
\DeclareMathOperator{\Ex}{\mathbb{E}}
\DeclareMathOperator{\rank}{rank}
\DeclareMathOperator{\erfc}{erfc}
\DeclareMathOperator{\erf}{erf}
\DeclareMathOperator{\cov}{cov}
\DeclareMathOperator{\cost}{cost}
\DeclareMathOperator{\comp}{comp}
\DeclareMathOperator{\corr}{corr}
\DeclareMathOperator{\diag}{diag}
\DeclareMathOperator{\var}{var}
\DeclareMathOperator{\opt}{opt}
\DeclareMathOperator{\brandnew}{new}
\DeclareMathOperator{\std}{std}
\DeclareMathOperator{\kurt}{kurt}
\DeclareMathOperator{\med}{med}
\DeclareMathOperator{\vol}{vol}
\DeclareMathOperator{\bias}{bias}
\DeclareMathOperator*{\argmax}{argmax}
\DeclareMathOperator*{\argmin}{argmin}
\DeclareMathOperator{\sign}{sign}
\DeclareMathOperator{\spann}{span}
\DeclareMathOperator{\cond}{cond}
\DeclareMathOperator{\trace}{trace}
\DeclareMathOperator{\Si}{Si}
%\DeclareMathOperator{\diag}{diag}
\DeclareMathOperator{\col}{col}
\DeclareMathOperator{\nullspace}{null}
\DeclareMathOperator{\Order}{{\mathcal O}}
%\DeclareMathOperator{\rank}{rank}

\newcommand{\vzero}{\bvec{0}}
\newcommand{\vone}{\bvec{1}}
\newcommand{\vinf}{\bvec{\infty}}
\newcommand{\va}{\bvec{a}}
\newcommand{\vA}{\bvec{A}}
\newcommand{\vb}{\bvec{b}}
\newcommand{\vB}{\bvec{B}}
\newcommand{\vc}{\bvec{c}}
\newcommand{\vd}{\bvec{d}}
\newcommand{\vD}{\bvec{D}}
\newcommand{\ve}{\bvec{e}}
\newcommand{\vf}{\bvec{f}}
\newcommand{\vF}{\bvec{F}}
\newcommand{\vg}{\bvec{g}}
\newcommand{\vG}{\bvec{G}}
\newcommand{\vh}{\bvec{h}}
\newcommand{\vi}{\bvec{i}}
\newcommand{\vj}{\bvec{j}}
\newcommand{\vk}{\bvec{k}}
\newcommand{\vK}{\bvec{K}}
\newcommand{\vl}{\bvec{l}}
\newcommand{\vell}{\bvec{\ell}}
\newcommand{\vL}{\bvec{L}}
\newcommand{\vm}{\bvec{m}}
\newcommand{\vp}{\bvec{p}}
\newcommand{\vq}{\bvec{q}}
\newcommand{\vr}{\bvec{r}}
\newcommand{\vs}{\bvec{s}}
\newcommand{\vS}{\bvec{S}}
\newcommand{\vt}{\bvec{t}}
\newcommand{\vT}{\bvec{T}}
\newcommand{\vu}{\bvec{u}}
\newcommand{\vU}{\bvec{U}}
\newcommand{\vv}{\bvec{v}}
\newcommand{\vV}{\bvec{V}}
\newcommand{\vw}{\bvec{w}}
\newcommand{\vW}{\bvec{W}}
\newcommand{\vx}{\bvec{x}}
\newcommand{\vX}{\bvec{X}}
\newcommand{\vy}{\bvec{y}}
\newcommand{\vY}{\bvec{Y}}
\newcommand{\vz}{\bvec{z}}
\newcommand{\vZ}{\bvec{Z}}

\newcommand{\ai}{\avec{\imath}}
\newcommand{\ak}{\avec{k}}
\newcommand{\avi}{\avec{\bvec{\imath}}}
\newcommand{\at}{\avec{t}}
\newcommand{\avt}{\avec{\vt}}
\newcommand{\ax}{\avec{x}}
\newcommand{\ah}{\avec{h}}
\newcommand{\akappa}{\avec{\kappa}}
\newcommand{\avx}{\avec{\vx}}
\newcommand{\ay}{\avec{y}}
\newcommand{\avy}{\avec{\vy}}
\newcommand{\avz}{\avec{\vz}}
\newcommand{\avzero}{\avec{\vzero}}
\newcommand{\aomega}{\avec{\omega}}
\newcommand{\avomega}{\avec{\vomega}}
\newcommand{\anu}{\avec{\nu}}
\newcommand{\avnu}{\avec{\vnu}}
\newcommand{\aDelta}{\avec{\Delta}}
\newcommand{\avDelta}{\avec{\vDelta}}

\newcommand{\valpha}{\bvec{\alpha}}
\newcommand{\vbeta}{\bvec{\beta}}
\newcommand{\vgamma}{\bvec{\gamma}}
\newcommand{\vGamma}{\bvec{\Gamma}}
\newcommand{\vdelta}{\bvec{\delta}}
\newcommand{\vDelta}{\bvec{\Delta}}
\newcommand{\vphi}{\bvec{\phi}}
\newcommand{\vvphi}{\bvec{\varphi}}
\newcommand{\vomega}{\bvec{\omega}}
\newcommand{\vlambda}{\bvec{\lambda}}
\newcommand{\vmu}{\bvec{\mu}}
\newcommand{\vnu}{\bvec{\nu}}
\newcommand{\vpsi}{\bvec{\psi}}
\newcommand{\vepsilon}{\bvec{\epsilon}}
\newcommand{\veps}{\bvec{\varepsilon}}
\newcommand{\veta}{\bvec{\eta}}
\newcommand{\vxi}{\bvec{\xi}}
\newcommand{\vtheta}{\bvec{\theta}}
\newcommand{\vtau}{\bvec{\tau}}
\newcommand{\vzeta}{\bvec{\zeta}}

\newcommand{\hA}{\widehat{A}}
\newcommand{\hvb}{\hat{\vb}}
\newcommand{\hcc}{\widehat{\cc}}
\newcommand{\hD}{\widehat{D}}
\newcommand{\hE}{\widehat{E}}
\newcommand{\hf}{\widehat{f}}
\newcommand{\hF}{\widehat{F}}
\newcommand{\hg}{\hat{g}}
\newcommand{\hvf}{\widehat{\bvec{f}}}
\newcommand{\hh}{\hat{h}}
\newcommand{\hH}{\widehat{H}}
\newcommand{\hi}{\hat{\imath}}
\newcommand{\hI}{\hat{I}}
\newcommand{\hci}{\widehat{\ci}}
\newcommand{\hj}{\hat{\jmath}}
\newcommand{\hp}{\hat{p}}
\newcommand{\hP}{\widehat{P}}
\newcommand{\hS}{\widehat{S}}
\newcommand{\hv}{\hat{v}}
\newcommand{\hV}{\widehat{V}}
\newcommand{\hx}{\hat{x}}
\newcommand{\hX}{\widehat{X}}
\newcommand{\hvX}{\widehat{\vX}}
\newcommand{\hy}{\hat{y}}
\newcommand{\hvy}{\hat{\vy}}
\newcommand{\hY}{\widehat{Y}}
\newcommand{\hvY}{\widehat{\vY}}
\newcommand{\hZ}{\widehat{Z}}
\newcommand{\hvZ}{\widehat{\vZ}}

\newcommand{\halpha}{\hat{\alpha}}
\newcommand{\hvalpha}{\hat{\valpha}}
\newcommand{\hbeta}{\hat{\beta}}
\newcommand{\hvbeta}{\hat{\vbeta}}
\newcommand{\hgamma}{\hat{\gamma}}
\newcommand{\hvgamma}{\hat{\vgamma}}
\newcommand{\hdelta}{\hat{\delta}}
\newcommand{\hvareps}{\hat{\varepsilon}}
\newcommand{\hveps}{\hat{\veps}}
\newcommand{\hmu}{\hat{\mu}}
\newcommand{\hnu}{\hat{\nu}}
\newcommand{\hvnu}{\widehat{\vnu}}
\newcommand{\homega}{\widehat{\omega}}
\newcommand{\hrho}{\hat{\rho}}
\newcommand{\hsigma}{\hat{\sigma}}
\newcommand{\htheta}{\hat{\theta}}
\newcommand{\hTheta}{\hat{\Theta}}
\newcommand{\htau}{\hat{\tau}}
\newcommand{\hxi}{\hat{\xi}}
\newcommand{\hvxi}{\hat{\vxi}}

\newcommand{\otau}{\overline{\tau}}
\newcommand{\oY}{\overline{Y}}

\newcommand{\rD}{\mathring{D}}
\newcommand{\rf}{\mathring{f}}
\newcommand{\rV}{\mathring{V}}

\newcommand{\ta}{\tilde{a}}
\newcommand{\tA}{\tilde{A}}
\newcommand{\tmA}{\widetilde{\mA}}
\newcommand{\tvb}{\tilde{\vb}}
\newcommand{\tcb}{\widetilde{\cb}}
\newcommand{\tB}{\widetilde{B}}
\newcommand{\tc}{\tilde{c}}
\newcommand{\tvc}{\tilde{\vc}}
\newcommand{\tfc}{\tilde{\fc}}
\newcommand{\tC}{\widetilde{C}}
\newcommand{\tcc}{\widetilde{\cc}}
\newcommand{\tD}{\widetilde{D}}
\newcommand{\te}{\tilde{e}}
\newcommand{\tE}{\widetilde{E}}
\newcommand{\tf}{\widetilde{f}}
\newcommand{\tF}{\widetilde{F}}
\newcommand{\tvf}{\tilde{\vf}}
\newcommand{\tcf}{\widetilde{\cf}}
\newcommand{\tg}{\tilde{g}}
\newcommand{\tG}{\widetilde{G}}
\newcommand{\tildeh}{\tilde{h}}
\newcommand{\tH}{\widetilde{H}}
\newcommand{\tch}{\widetilde{\ch}}
\newcommand{\tK}{\widetilde{K}}
\newcommand{\tvk}{\tilde{\vk}}
\newcommand{\tM}{\widetilde{M}}
\newcommand{\tn}{\tilde{n}}
\newcommand{\tN}{\widetilde{N}}
\newcommand{\tQ}{\widetilde{Q}}
\newcommand{\tR}{\widetilde{R}}
\newcommand{\tS}{\widetilde{S}}
\newcommand{\tvS}{\widetilde{\vS}}
\newcommand{\tT}{\widetilde{T}}
\newcommand{\tv}{\tilde{v}}
\newcommand{\tV}{\widetilde{V}}
\newcommand{\tvx}{\tilde{\vx}}
\newcommand{\tW}{\widetilde{W}}
\newcommand{\tx}{\tilde{x}}
\newcommand{\tX}{\widetilde{X}}
\newcommand{\tvX}{\widetilde{\vX}}
\newcommand{\ty}{\tilde{y}}
\newcommand{\tvy}{\tilde{\vy}}
\newcommand{\tz}{\tilde{z}}
\newcommand{\tZ}{\widetilde{Z}}
\newcommand{\tL}{\widetilde{L}}
\newcommand{\tP}{\widetilde{P}}
\newcommand{\tY}{\widetilde{Y}}
\newcommand{\tmH}{\widetilde{\mH}}
\newcommand{\tmK}{\widetilde{\mK}}
\newcommand{\tmM}{\widetilde{\mM}}
\newcommand{\tmQ}{\widetilde{\mQ}}
\newcommand{\tct}{\widetilde{\ct}}
\newcommand{\talpha}{\tilde{\alpha}}
\newcommand{\tdelta}{\tilde{\delta}}
\newcommand{\tDelta}{\tilde{\Delta}}
\newcommand{\tvareps}{\tilde{\varepsilon}}
\newcommand{\tveps}{\tilde{\veps}}
\newcommand{\tlambda}{\tilde{\lambda}}
\newcommand{\tmu}{\tilde{\mu}}
\newcommand{\tnu}{\tilde{\nu}}
\newcommand{\trho}{\tilde{\rho}}
\newcommand{\tvarrho}{\tilde{\varrho}}
\newcommand{\ttheta}{\tilde{\theta}}
\newcommand{\tsigma}{\tilde{\sigma}}
\newcommand{\tvmu}{\tilde{\vmu}}
\newcommand{\tphi}{\tilde{\phi}}
\newcommand{\tPhi}{\widetilde{\Phi}}
\newcommand{\tvphi}{\tilde{\vphi}}
\newcommand{\ttau}{\tilde{\tau}}
\newcommand{\txi}{\tilde{\xi}}
\newcommand{\tvxi}{\tilde{\vxi}}


\newcommand{\mA}{\mathsf{A}}
\newcommand{\mB}{\mathsf{B}}
\newcommand{\mC}{\mathsf{C}}
\newcommand{\vmC}{\bvec{\mC}}
\newcommand{\mD}{\mathsf{D}}
\newcommand{\mF}{\mathsf{F}}
\newcommand{\mG}{\mathsf{G}}
\newcommand{\mH}{\mathsf{H}}
\newcommand{\mI}{\mathsf{I}}
\newcommand{\mK}{\mathsf{K}}
\newcommand{\mL}{\mathsf{L}}
\newcommand{\mM}{\mathsf{M}}
\newcommand{\mP}{\mathsf{P}}
\newcommand{\mQ}{\mathsf{Q}}
\newcommand{\mR}{\mathsf{R}}
\newcommand{\mS}{\mathsf{S}}
\newcommand{\mT}{\mathsf{T}}
\newcommand{\mU}{\mathsf{U}}
\newcommand{\mV}{\mathsf{V}}
\newcommand{\mW}{\mathsf{W}}
\newcommand{\mX}{\mathsf{X}}
\newcommand{\mLambda}{\UpSfLambda}
\newcommand{\mSigma}{\UpSfSigma}
\newcommand{\mzero}{\mathsf{0}}
\newcommand{\mGamma}{\mathsf{\Gamma}}

\newcommand{\bbE}{\mathbb{E}}
\newcommand{\bbF}{\mathbb{F}}
\newcommand{\bbK}{\mathbb{K}}
\newcommand{\bbV}{\mathbb{V}}
\newcommand{\bbZ}{\mathbb{Z}}
\newcommand{\bbone}{\mathbbm{1}}
\newcommand{\naturals}{\mathbb{N}}
\newcommand{\reals}{\mathbb{R}}
\newcommand{\integers}{\mathbb{Z}}
\newcommand{\natzero}{\mathbb{N}_{0}}
\newcommand{\rationals}{\mathbb{Q}}
\newcommand{\complex}{\mathbb{C}}

\newcommand{\ca}{\mathcal{A}}
\newcommand{\cb}{\mathcal{B}}
\providecommand{\cc}{\mathcal{C}}
\newcommand{\cd}{\mathcal{D}}
\newcommand{\cf}{\mathcal{F}}
\newcommand{\cg}{\mathcal{G}}
\newcommand{\ch}{\mathcal{H}}
\newcommand{\ci}{\mathcal{I}}
\newcommand{\cj}{\mathcal{J}}
\newcommand{\ck}{\mathcal{K}}
\newcommand{\cl}{\mathcal{L}}
\newcommand{\cm}{\mathcal{M}}
\newcommand{\tcm}{\widetilde{\cm}}
\newcommand{\cn}{\mathcal{N}}
\newcommand{\cp}{\mathcal{P}}
\newcommand{\calr}{\mathcal{R}}
\newcommand{\cs}{\mathcal{S}}
\newcommand{\ct}{\mathcal{T}}
\newcommand{\cu}{\mathcal{U}}
\newcommand{\cv}{\mathcal{V}}
\newcommand{\cw}{\mathcal{W}}
\newcommand{\cx}{\mathcal{X}}
\newcommand{\tcx}{\widetilde{\cx}}
\newcommand{\cy}{\mathcal{Y}}
\newcommand{\cz}{\mathcal{Z}}

\newcommand{\fc}{\mathfrak{c}}
\newcommand{\fC}{\mathfrak{C}}
\newcommand{\fh}{\mathfrak{h}}
\newcommand{\fu}{\mathfrak{u}}

\newcommand{\me}{\ensuremath{\mathrm{e}}} % for math number 'e', 2.718 281 8..., tha base of natural logarithms
\newcommand{\mi}{\ensuremath{\mathrm{i}}} % for math number 'i', the imaginary unit
\newcommand{\mpi}{\ensuremath{\mathrm{\pi}}} % for math number 'pi', the circumference of a circle of diameter 1


\title{Generalizations of Lattices and Nets}
\author{Fred J. Hickernell}
\author{Peter Kritzer}

\begin{document}
\maketitle

\section{Basics}

Define the $2$-adic map from the non-negative integers to the unit interval:
\begin{equation}
\phi(i) = \phi(i_0 + 2 i_1 + 4 i_2 + \cdots) = \frac{i_0}{2} + \frac{i_1}{4} + \frac{i_2} {8} + \cdots, \qquad i \in \natzero
\end{equation}
We will define a group labeled by $\natzero$, and then map these points back to the unit interval via $\phi$.

The addition operator, $\oplus$, is defined iteratively. For all $m \in \natzero$, let 
\begin{align*}
    \ci_m & = \{0, \ldots, 2^m -1\}, \\
    \cj_m & = \{2^m, \dots, 2^{m+1}-1 \}, \\
    \pi:\naturals \to \naturals & \text{ be a bijective map satisfying } \pi(\cj_m) = \cj_m \quad \forall m \in \natzero, \\
    \pi_m(i) & = \pi(i+2^m) \quad \forall i \in \ci_m, \qquad \pi_m:\ci_m \to \cj_m.
\end{align*}
Also, let $0 \oplus 0 = 0$.  Then assuming that $\ci_m$ is already an Abelian group under $\oplus$, which it is for $m=0$, the definition of $\oplus$ for $\ci_m$ is extended to $\ci_{m+1} = \ci_m \cup \cj_m$ as follows:
\begin{subequations} \label{oplusDef}
\begin{gather}
    \label{oplusDefA}
    i \oplus j = \pi_m( i \oplus \pi_m^{-1}(j)), \qquad \forall i \in \ci_m,\ j \in \cj_m, \\
    \label{oplusDefB}
    j_1 \oplus j_2 = \pi_m^{-1}( j_1) \oplus \pi_m^{-1}(j_2) \ominus \pi_m^{-1}(2^m), \qquad \forall j_1, j_2 \in \cj_m.
\end{gather}
\end{subequations}
The inverse of $i$ is denoted $\ominus i$.  Moreover, For shorthand $i \ominus j$ means $i \oplus (\ominus j)$ for all $i, j \in \ci_m$.

We claim that $\ci_{m+1}$ is now also an Abelian group under $\oplus$.  It is trivial to check that $\ci_{m+1}$ is closed under $\oplus$ and that commutativity holds on $\ci_{m+1}$.  Moreover, by \eqref{oplusDefA}, $0$ remains the identity element.  The additive inverses of elements in $\cj_m$ are 
\begin{equation}
    \ominus j = \pi_m\bigl (\pi_m^{-1}(2^m) \ominus \pi_m^{-1}(j) \bigr ) , \qquad \forall j \in \cj_m,
\end{equation}
since 
\[
j \oplus \pi_m\bigl (\pi_m^{-1}(2^m) \ominus \pi_m^{-1}(j)  \bigr ) = \pi_m^{-1}( j) \oplus \bigl ( \pi_m^{-1}(2^m) \ominus \pi_m^{-1}(j) \bigr ) \ominus \pi_m^{-1}(2^m) = 0
\]
by  \eqref{oplusDefB} and the associative property on $\ci_m$.  The associative property for $\ci_{m+1}$ is proven in Lemma \ref{lem:assocProp}.



Under definition \eqref{oplusDefB} it follows that 
\begin{equation*}
    j = \pi_m(j \oplus 2^m),  \qquad \forall j \in \cj_m.
\end{equation*}
Thus, if $\oplus$ is defined first, then $\pi_m$ can be inferred from the equation above.



\section{Examples}

This first example corresponds to $i \oplus j$ defined by digit-wise addition:
\[
\begin{array}{ccc|cccccccc}
&& \multicolumn{9}{c}{i \oplus j} \\
 \pi(i) & \phi(i) & \multicolumn{1}{c}{i\backslash j} & 0 & 1 & 2 & 3 & 4 & 5 & 6 & 7 \\ 
    \toprule
      & 0 & 0 & 0\\
    1 & 0.5 & 1 & 1 & 0 \\
    2 & 0.25  & 2 & 2 & 3 & 0\\
    3 & 0.75  & 3 & 3 & 2 & 1 & 0\\
    4 & 0.125 & 4 & 4 & 5 & 6 & 7 & 0\\
    5 & 0.625 & 5 & 5 & 4 & 7 & 6 & 1 & 0\\
    6 & 0.375 & 6 & 6 & 7 & 4 & 5 & 2 & 3 & 0\\
    7 & 0.875 & 7 & 7 & 6 & 5 & 4 & 3 & 2 & 1 & 0\\
\end{array}
\]
A second example corresponds to $\phi(i \oplus j) = \phi(i) + \phi(j) \bmod 1$:
\[
\begin{array}{ccc|ccccccccc}
&& \multicolumn{9}{c}{i \oplus j} \\
 \pi(i) & \phi(i) & \multicolumn{1}{c}{i\backslash j} & 0 & 1 & 2 & 3 & 4 & 5 & 6 & 7 \\ 
    \toprule
      & 0 &     0 & 0\\
    1 & 0.5   & 1 & 1 & 0 \\
    3 & 0.25  & 2 & 2 & 3 & 1\\
    2 & 0.75  & 3 & 3 & 2 & 0 & 1\\
    7 & 0.125 & 4 & 4 & 5 & 6 & 7 & 2\\
    6 & 0.625 & 5 & 5 & 4 & 7 & 6 & 3 & 2\\
    4 & 0.375 & 6 & 6 & 7 & 5 & 4 & 1 & 0 & 3\\
    5 & 0.875 & 7 & 7 & 6 & 4 & 5 & 0 & 1 & 2 & 3\\
\end{array}
\]


\section{Appendix}
\begin{lemma} \label{lem:assocProp}
If $oplus$ is defined iteratively as in \eqref{oplusDef}, then the associative property holds for $\ci_{m+1}$, provided that it holds for $\ci_m$.
\end{lemma}  
\begin{proof}
All that is needed is to check the associative property when at least one of the elements is in $\cj_m = \ci_{m+1} \setminus \ci_m$.  For all $i_1, i_2 \in \ci_m$ and $j_1, j_2, j_3 \in \cj_m$, 
\begin{align*}
    (i_1 \oplus i_2) \oplus j_3 & = \pi_m((i_1 \oplus i_2) \oplus \pi_m^{-1}(j_3)) && \text{by } \eqref{oplusDefA} \\
    & = \pi_m(i_1 \oplus (i_2 \oplus \pi_m^{-1}(j_3))) && \text{by the associative property on } \ci_m \\
    & = \pi_m(i_1 \oplus \pi_m^{-1}(\pi_m(i_2 \oplus \pi_m^{-1}(j_3)))) && \text{since $\pi_m$ is bijective}\\
    & = i_1 \oplus \pi_m(i_2 \oplus \pi_m^{-1}(j_3)) && \text{by } \eqref{oplusDefA} \\
    & = i_1 \oplus (i_2 \oplus j_3) && \text{by } \eqref{oplusDefA}, \\
    (i_1 \oplus j_2) \oplus j_3 & = \pi_m \bigl(i_1 \oplus \pi_m^{-1}(j_2) \bigr) \oplus j_3 && \text{by } \eqref{oplusDefA} \\
    & = \bigl (i_1 \oplus \pi_m^{-1}(j_2) \bigr) \oplus \pi_m^{-1}(j_3) \ominus \pi_m^{-1}(2^m) && \text{by } \eqref{oplusDefB} \\
    & = i_1 \oplus \bigl (\pi_m^{-1}(j_2) \oplus \pi_m^{-1}(j_3)\ominus \pi_m^{-1}(2^m) \bigr) && \text{by the associative property on } \ci_m \\
    & = i_1 \oplus (j_2 \oplus j_3) && \text{by } \eqref{oplusDefB}, \\
    (j_1 \oplus j_2) \oplus j_3 & = \bigl(\pi_m^{-1}(j_1) \oplus \pi_m^{-1}(j_2) \ominus \pi_m^{-1}(2^m) \bigr) \oplus j_3 && \text{by } \eqref{oplusDefB} \\
    & = \pi_m\bigl ((\pi_m^{-1}(j_1) \oplus \pi_m^{-1}(j_2)  \ominus \pi_m^{-1}(2^m)) \\
    &\qquad \qquad \qquad \qquad \oplus \pi_m^{-1}(j_3) \bigr) && \text{by } \eqref{oplusDefA} \\
    & = \pi_m\bigl (\pi_m^{-1}(j_1) \\
    & \qquad  \oplus (\pi_m^{-1}(j_2) \oplus \pi_m^{-1}(j_3)\ominus \pi_m^{-1}(2^m)) \bigr) && \text{by the associative property on } \ci_m \\
    & = j_1 \oplus \bigl (\pi_m^{-1}(j_2) \oplus \pi_m^{-1}(j_3)) \ominus \pi_m^{-1}(2^m) \bigr) && \text{by } \eqref{oplusDefA} \\
    & = j_1 \oplus (j_2 \oplus j_3) && \text{by } \eqref{oplusDefB}.
\end{align*}
Thus, the associative property holds.
\end{proof}

\end{document}