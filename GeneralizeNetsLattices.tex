\documentclass[12pt]{amsart}

\hoffset -0.7in
\textwidth 6.5in

%% Language and font encodings
\usepackage[english]{babel}
\usepackage[utf8x]{inputenc}
\usepackage[T1]{fontenc}


%% Useful packages
\usepackage{amsmath,booktabs}
\usepackage{graphicx}
\newtheorem{lemma}{Lemma}

\input{FJHDef.tex}

\title{Generalizations of Lattices and Nets}
\author{Fred J. Hickernell}
\author{Peter Kritzer}

\begin{document}
\maketitle

\section{Introduction}

\section{Overview}

\subsection{Cubature Rules}
Define the $2$-adic map from the non-negative integers to the unit interval:
\begin{equation} \label{phiDef}
\phi(i) = \phi(i_0 + 2 i_1 + 4 i_2 + \cdots) = \frac{i_0}{2} + \frac{i_1}{4} + \frac{i_2} {8} + \cdots, \qquad \forall i = i_0 + 2 i_1 + \cdots \in \natzero.
\end{equation}
Then $\cx:= \phi(\natzero)$ is the set of all binary fractions in $[0,1)$.


In Section \ref{sec:Group} we define an Abelian group $(\natzero,\oplus)$ and with subgroups
\begin{equation} \label{ImDef}
    (\ci_m,\oplus), \quad \ci_m := \{0, \ldots, 2^m -1\}, \qquad \forall m \in \natzero.
\end{equation}
See ??? for examples of $\oplus$ defined on $\ci_3$.  The definition of  $(\natzero,\oplus)$ is  extended naturally to  $(\natzero^d, \oplus)$ by defining $\oplus$ on vectors to act component-wise.  The map $\phi$ is  extended to vector inputs as $\vphi(\vi) := \vphi(i_1, \ldots, i_d) := (\phi(i_1), \ldots, \phi(i_d))$, and $\vphi(\natzero^d) = \cx^d$. 

Our goal is to define subgroups, $(\cp,\oplus)$, of $(\natzero^d, \oplus)$ that yield good cubature rules of the form
\begin{equation*}
    \int_{[0,1)^d} f(\vx) \, \dif \vx \approx \frac 1{\abs{\cp}} \sum_{\vi \in \cp} f(\vphi(\vi \oplus \vj)),
\end{equation*}
where $\vj$ is some fixed shift.  The aim is to develop quality criteria for $\cp$ that bound the cubature error, and to show that these quality criteria decay at favorable rates as $\abs{\cp} \to \infty$.

\subsection{Worst-Case Cubature Error}
The error analysis proceeds by identifying functions, $\psi_{\vk}:\natzero^d \to \reals$ for all $\vk \in \ck$ that are the characters of the group $(\natzero^d, \oplus)$, i.e., 
\begin{equation} \label{charDef}
    \psi_{\vk}(\vi \oplus \vj) = \psi_{\vk}(\vi) \psi_{\vk}(\vj), \quad \abs{\psi_{\vk}(\vi)} = 1, \qquad \forall \vi, \vj \in \natzero^d.
\end{equation}
By convention, $\psi_{\vzero} = 1$.  Let $\Psi_{\vk}:[0,1)^d \to \reals$ be defined by 
\begin{equation} \label{PsiDef}
    \Psi_{\vk}(\vx) = \lim_{\ell \to \infty}  \psi_{\vk}(\vphi^{-1}(\vx_\ell)), 
\end{equation}
where $\vx_\ell \in \cx^d$ is the proper binary expansion of $\vx$ truncated at $\ell$ digits in each coordinate.  This implies that $\Psi_{\vk}$ can only have discontinuities at points in $\cx$.  Note that $\Psi_{\vzero} = 1$.

The space of integrands is defined to be 
\begin{equation} \label{FSpaceDef}
    \cf : = \left \{ f = \sum_{\vk} \hf_{\vk} \Psi_{\vk} : \norm[\cf]{f}^2 := \sum_{\vk} \abs{\frac{\hf_{\vk}}{w_{\vk}}}^2 < \infty   \right \}.
\end{equation}
Here the weights, $w_{\vk}$, are chosen to satisfy
\begin{equation}
    \sum_{\vk} \abs{w_{\vk} \Psi_{\vk}(\vx)}^2 <  \infty \qquad \forall \vx \in [0,1)^d,
\end{equation}
thus ensuring that function evaluation at any particular point is a bounded, linear functional for $\cf$.

It is shown that $\int_{[0,1)^d} \Psi_{\vk}(\vx) \, \dif \vx = \delta_{\vk,\vzero}$.  Employing the series expression for the integrand implies that 
\begin{align}
\label{errA}
    \int_{[0,1)^d} f(\vx) \, \dif \vx - \frac 1{\abs{\cp}} \sum_{\vi \in \cp} f(\vphi(\vi \oplus \vj)) 
    & = - \sum_{\vk \in \ck} \hf_{\vk} \times \frac 1{\abs{\cp}} \sum_{\vi \in \cp} \Psi_{\vk}(\vphi(\vi \oplus \vj))   \qquad \text{by }\eqref{FSpaceDef} \\
    \nonumber
    & = - \sum_{\vk \in \ck} \hf_{\vk}  \times \frac 1{\abs{\cp}} \sum_{\vi \in \cp} \psi_{\vk}(\vi \oplus \vj) 
    \qquad \text{by }\eqref{PsiDef}\\
    \nonumber 
    & = - \sum_{\vk \in \ck} \hf_{\vk} \psi_{\vk}(\vj) \times  \frac 1{\abs{\cp}} \sum_{\vi \in \cp} \psi_{\vk}(\vi) \qquad \text{by }  \eqref{charDef} .
\end{align}
The last term on the right can take on one of two values.  The subgroup structure of $\cp$ and \eqref{charDef} implies that 
\begin{equation*}
    \frac 1{\abs{\cp}} \sum_{\vi \in \cp} \psi_{\vk}(\vi) 
    = \frac 1{\abs{\cp}} \sum_{\vi \in \cp} \psi_{\vk}(\vi \oplus \vi') 
 = \psi_{\vk}(\vi') \frac 1{\abs{\cp}} \sum_{\vi \in \cp} \psi_{\vk}(\vi) \qquad  \forall \vi' \in \cp.
\end{equation*}
and so
\begin{equation*}
    \left [ 1 - \psi_{\vk}(\vi') \right ] \frac 1{\abs{\cp}} \sum_{\vi \in \cp} \psi_{\vk}(\vi) 
    = 0 \qquad  \forall \vi' \in \cp.
\end{equation*}
Define the \emph{dual set} of the subgroup $\cp$ by 
\begin{equation}
    \cp^{\perp} : = \{\vk \in \ck : \psi_{\vk}(\vi) = 1 \ \forall \vi \in \cp \}.
\end{equation}
Then the sum at the end of \eqref{errA} is either one or zero, depending on whether or not $\vk$ is in the dual set:
\begin{equation}
    \frac 1{\abs{\cp}} \sum_{\vi \in \cp} \psi_{\vk}(\vi) = 
    \begin{cases} 1, & \vk \in \cp^{\perp}, \\
    0, & \vk \notin \cp^{\perp}.
    \end{cases}
\end{equation}
Furthermore,  from  \eqref{charDef} and \eqref{errA} the cubature error is bounded by
\begin{equation}
    \abs{\int_{[0,1)^d} f(\vx) \, \dif \vx - \frac 1{\abs{\cp}} \sum_{\vi \in \cp} f(\vphi(\vi \oplus \vj))}  
    \le \sum_{\vk \in \cp^\perp} \abs{\hf_{\vk}} \le \norm[2]{\bigl(w_{\vk} \bigr)_{\vk \in \cp^\perp} } \norm[\cf]{f}
\end{equation}

The quantity $\norm[2]{\bigl(w_{\vk} \bigr)_{\vk \in \cp^\perp} }$ corresponds to the worst-case error for the cubature rule.

\subsection{Randomized Cubature Error}
The cubature rule may be randomized by applying some randomization to the set $\cp$.  We consider randomizations of the form $\veta : \natzero^d \to \natzero^d$, which satisfy 
\begin{equation}
    \veta(\vi) \sim \cu(\natzero^d).
\end{equation}
This implies that 



\section{The Group Structure} \label{sec:Group}
The addition operator, $\oplus$, is defined iteratively. For all $m \in \natzero$, define 
\begin{align*}
    \cj_m & := \{2^m, \dots, 2^{m+1}-1 \} = \ci_{m+1} \setminus \ci_m, \\
    \pi & : \naturals \to \naturals \text{ to be be a bijective map satisfying } \pi(\cj_m) = \cj_m \quad \forall m \in \natzero, \\
    \pi_m & : \begin{cases} 
    \ci_m \to \cj_m, \\
    i \mapsto \pi(i+2^m).
    \end{cases}
\end{align*}
Also, let $0 \oplus 0 = 0$.  Then assuming that $\ci_m$ is already an Abelian group under $\oplus$, which it is for $m=0$, the definition of $\oplus$ for $\ci_m$ is extended to $\ci_{m+1} = \ci_m \cup \cj_m$ as follows:
\begin{subequations} \label{oplusDef}
\begin{gather}
    \label{oplusDefA}
    i \oplus j = \pi_m( i \oplus \pi_m^{-1}(j)), \qquad \forall i \in \ci_m,\ j \in \cj_m, \\
    \label{oplusDefB}
    j_1 \oplus j_2 = \pi_m^{-1}( j_1) \oplus \pi_m^{-1}(j_2) \ominus \pi_m^{-1}(2^m), \qquad \forall j_1, j_2 \in \cj_m.
\end{gather}
\end{subequations}
The inverse of $i$ is denoted $\ominus i$.  Moreover, For shorthand $i \ominus j$ means $i \oplus (\ominus j)$ for all $i, j \in \ci_m$.

We claim that $\ci_{m+1}$ is now also an Abelian group under $\oplus$.  It is trivial to check that $\ci_{m+1}$ is closed under $\oplus$ and that commutativity holds on $\ci_{m+1}$.  Moreover, by \eqref{oplusDefA}, $0$ remains the identity element.  The additive inverses of elements in $\cj_m$ are 
\begin{equation}
    \ominus j = \pi_m\bigl (\pi_m^{-1}(2^m) \ominus \pi_m^{-1}(j) \bigr ) , \qquad \forall j \in \cj_m,
\end{equation}
since 
\[
j \oplus \pi_m\bigl (\pi_m^{-1}(2^m) \ominus \pi_m^{-1}(j)  \bigr ) = \pi_m^{-1}( j) \oplus \bigl ( \pi_m^{-1}(2^m) \ominus \pi_m^{-1}(j) \bigr ) \ominus \pi_m^{-1}(2^m) = 0
\]
by  \eqref{oplusDefB} and the associative property on $\ci_m$.  The associative property for $\ci_{m+1}$ is proven in Lemma \ref{lem:assocProp}.



Under definition \eqref{oplusDefB} it follows that 
\begin{equation*}
    j = \pi_m(j \oplus 2^m),  \qquad \forall j \in \cj_m.
\end{equation*}
Thus, if $\oplus$ is defined first, then $\pi_m$ can be inferred from the equation above.



\section{Examples}

This first example corresponds to $i \oplus j$ defined by digit-wise addition:
\[
\begin{array}{ccc|cccccccc}
&& \multicolumn{9}{c}{i \oplus j} \\
 \pi(i) & \phi(i) & \multicolumn{1}{c}{i\backslash j} & 0 & 1 & 2 & 3 & 4 & 5 & 6 & 7 \\ 
    \toprule
      & 0 & 0 & 0\\
    1 & 0.5 & 1 & 1 & 0 \\
    2 & 0.25  & 2 & 2 & 3 & 0\\
    3 & 0.75  & 3 & 3 & 2 & 1 & 0\\
    4 & 0.125 & 4 & 4 & 5 & 6 & 7 & 0\\
    5 & 0.625 & 5 & 5 & 4 & 7 & 6 & 1 & 0\\
    6 & 0.375 & 6 & 6 & 7 & 4 & 5 & 2 & 3 & 0\\
    7 & 0.875 & 7 & 7 & 6 & 5 & 4 & 3 & 2 & 1 & 0\\
\end{array}
\]
A second example corresponds to $\phi(i \oplus j) = \phi(i) + \phi(j) \bmod 1$:
\[
\begin{array}{ccc|ccccccccc}
&& \multicolumn{9}{c}{i \oplus j} \\
 \pi(i) & \phi(i) & \multicolumn{1}{c}{i\backslash j} & 0 & 1 & 2 & 3 & 4 & 5 & 6 & 7 \\ 
    \toprule
      & 0 &     0 & 0\\
    1 & 0.5   & 1 & 1 & 0 \\
    3 & 0.25  & 2 & 2 & 3 & 1\\
    2 & 0.75  & 3 & 3 & 2 & 0 & 1\\
    7 & 0.125 & 4 & 4 & 5 & 6 & 7 & 2\\
    6 & 0.625 & 5 & 5 & 4 & 7 & 6 & 3 & 2\\
    4 & 0.375 & 6 & 6 & 7 & 5 & 4 & 1 & 0 & 3\\
    5 & 0.875 & 7 & 7 & 6 & 4 & 5 & 0 & 1 & 2 & 3\\
\end{array}
\]


\section{Appendix}
\begin{lemma} \label{lem:assocProp}
If $oplus$ is defined iteratively as in \eqref{oplusDef}, then the associative property holds for $\ci_{m+1}$, provided that it holds for $\ci_m$.
\end{lemma}  
\begin{proof}
All that is needed is to check the associative property when at least one of the elements is in $\cj_m = \ci_{m+1} \setminus \ci_m$.  For all $i_1, i_2 \in \ci_m$ and $j_1, j_2, j_3 \in \cj_m$, 
\begin{align*}
    (i_1 \oplus i_2) \oplus j_3 & = \pi_m((i_1 \oplus i_2) \oplus \pi_m^{-1}(j_3)) && \text{by } \eqref{oplusDefA} \\
    & = \pi_m(i_1 \oplus (i_2 \oplus \pi_m^{-1}(j_3))) && \text{by the associative property on } \ci_m \\
    & = \pi_m(i_1 \oplus \pi_m^{-1}(\pi_m(i_2 \oplus \pi_m^{-1}(j_3)))) && \text{since $\pi_m$ is bijective}\\
    & = i_1 \oplus \pi_m(i_2 \oplus \pi_m^{-1}(j_3)) && \text{by } \eqref{oplusDefA} \\
    & = i_1 \oplus (i_2 \oplus j_3) && \text{by } \eqref{oplusDefA}, \\
    (i_1 \oplus j_2) \oplus j_3 & = \pi_m \bigl(i_1 \oplus \pi_m^{-1}(j_2) \bigr) \oplus j_3 && \text{by } \eqref{oplusDefA} \\
    & = \bigl (i_1 \oplus \pi_m^{-1}(j_2) \bigr) \oplus \pi_m^{-1}(j_3) \ominus \pi_m^{-1}(2^m) && \text{by } \eqref{oplusDefB} \\
    & = i_1 \oplus \bigl (\pi_m^{-1}(j_2) \oplus \pi_m^{-1}(j_3)\ominus \pi_m^{-1}(2^m) \bigr) && \text{by the associative property on } \ci_m \\
    & = i_1 \oplus (j_2 \oplus j_3) && \text{by } \eqref{oplusDefB}, \\
    (j_1 \oplus j_2) \oplus j_3 & = \bigl(\pi_m^{-1}(j_1) \oplus \pi_m^{-1}(j_2) \ominus \pi_m^{-1}(2^m) \bigr) \oplus j_3 && \text{by } \eqref{oplusDefB} \\
    & = \pi_m\bigl ((\pi_m^{-1}(j_1) \oplus \pi_m^{-1}(j_2)  \ominus \pi_m^{-1}(2^m)) \\
    &\qquad \qquad \qquad \qquad \oplus \pi_m^{-1}(j_3) \bigr) && \text{by } \eqref{oplusDefA} \\
    & = \pi_m\bigl (\pi_m^{-1}(j_1) \\
    & \qquad  \oplus (\pi_m^{-1}(j_2) \oplus \pi_m^{-1}(j_3)\ominus \pi_m^{-1}(2^m)) \bigr) && \text{by the associative property on } \ci_m \\
    & = j_1 \oplus \bigl (\pi_m^{-1}(j_2) \oplus \pi_m^{-1}(j_3)) \ominus \pi_m^{-1}(2^m) \bigr) && \text{by } \eqref{oplusDefA} \\
    & = j_1 \oplus (j_2 \oplus j_3) && \text{by } \eqref{oplusDefB}.
\end{align*}
Thus, the associative property holds.
\end{proof}

\end{document}