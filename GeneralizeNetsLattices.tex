\documentclass[12pt]{amsart}

\hoffset -0.7in
\textwidth 6.5in

%% Language and font encodings
\usepackage[english]{babel}
\usepackage[utf8x]{inputenc}
\usepackage[T1]{fontenc}


%% Useful packages
\usepackage{amsmath,booktabs}
\usepackage{graphicx}

\input{FJHDef.tex}

\title{Generalizations of Lattices and Nets}
\author{Fred J. Hickernell}
\author{Peter Kritzer}

\begin{document}
\maketitle

\section{Basics}

Define the $2$-adic map from the non-negative integers to the unit interval:
\begin{equation}
\phi(i) = \phi(i_0 + 2 i_1 + 4 i_2 + \cdots) = \frac{i_0}{2} + \frac{i_1}{4} + \frac{i_2} {8} + \cdots, \qquad i \in \natzero
\end{equation}
We will define a group labeled by $\natzero$, and then map these points back to the unit interval.

The addition operator, $\oplus$, is defined iteratively. For all $m \in \natzero$, let $\ci_m = \{0, \ldots, 2^m -1\}$, $\cj_m = \{2^m, \dots, 2^{m+1}\}$, and let $\pi_m:\ci_m \to \cj_m$ be a bijective map.  Also, let $0 \oplus 0 = 0$.  Then assuming that $\ci_m$ is already an Abelian group under $\oplus$, which it is for $m=0$, the definition of $\oplus$ for $\ci_m$ is extended to $\ci_{m+1} = \ci_m \cup \cj_m$ as follows:
\begin{subequations} \label{oplusDef}
\begin{gather}
    \label{oplusDefA}
    i \oplus j = \pi_m( i \oplus \pi_m^{-1}(j)), \qquad \forall i \in \ci_m,\ j \in \cj_m, \\
    \label{oplusDefB}
    j_1 \oplus j_2 = \pi_m^{-1}( j_1) \oplus \pi_m^{-1}(j_2) \ominus \pi_m^{-1}(2^m), \qquad \forall j_1, j_2 \in \cj_m.
\end{gather}
\end{subequations}
The inverse of $i$ is denoted $\ominus i$.  Moreover, For shorthand $i \ominus j$ means $i \oplus (\ominus j)$ for all $i, j \in \ci_m$.

We claim that $\ci_{m+1}$ is now also an Abelian group under $\oplus$.  It is trivial to check that $\ci_{m+1}$ is closed under $\oplus$ and that commutativity holds on $\ci_{m+1}$.  Moreover, by \eqref{oplusDefA}, $0$ remains the identity element.  The additive inverses of elements in $\cj_m$ are 
\begin{equation}
    \ominus j = \pi_m\bigl (\pi_m^{-1}(2^m) \ominus \pi_m^{-1}(j) \bigr ) , \qquad \forall j \in \cj_m,
\end{equation}
since 
\[
j \oplus \pi_m\bigl (\pi_m^{-1}(2^m) \ominus \pi_m^{-1}(j)  \bigr ) = \pi_m^{-1}( j) \oplus \bigl ( \pi_m^{-1}(2^m) \ominus \pi_m^{-1}(j) \bigr ) \ominus \pi_m^{-1}(2^m) = 0
\]
by  \eqref{oplusDefB} and the associative property on $\ci_m$.  

All that remains is to check the associative property when at least one of the elements is in $\cj_m$.  For all $i_1, i_2 \in \ci_m$ and $j_1, j_2, j_3 \in \cj_m$, 
\begin{align*}
    (i_1 \oplus i_2) \oplus j_3 & = \pi_m((i_1 \oplus i_2) \oplus \pi_m^{-1}(j_3)) && \text{by } \eqref{oplusDefA} \\
    & = \pi_m(i_1 \oplus (i_2 \oplus \pi_m^{-1}(j_3))) && \text{by the associative property on } \ci_m \\
    & = \pi_m(i_1 \oplus \pi_m^{-1}(\pi_m(i_2 \oplus \pi_m^{-1}(j_3)))) && \text{since $\pi_m$ is bijective}\\
    & = i_1 \oplus \pi_m(i_2 \oplus \pi_m^{-1}(j_3)) && \text{by } \eqref{oplusDefA} \\
    & = i_1 \oplus (i_2 \oplus j_3) && \text{by } \eqref{oplusDefA}, \\
    (i_1 \oplus j_2) \oplus j_3 & = \pi_m \bigl(i_1 \oplus \pi_m^{-1}(j_2) \bigr) \oplus j_3 && \text{by } \eqref{oplusDefA} \\
    & = \bigl (i_1 \oplus \pi_m^{-1}(j_2) \bigr) \oplus \pi_m^{-1}(j_3) \ominus \pi_m^{-1}(2^m) && \text{by } \eqref{oplusDefB} \\
    & = i_1 \oplus \bigl (\pi_m^{-1}(j_2) \oplus \pi_m^{-1}(j_3)\ominus \pi_m^{-1}(2^m) \bigr) && \text{by the associative property on } \ci_m \\
    & = i_1 \oplus (j_2 \oplus j_3) && \text{by } \eqref{oplusDefB}, \\
    (j_1 \oplus j_2) \oplus j_3 & = \bigl(\pi_m^{-1}(j_1) \oplus \pi_m^{-1}(j_2) \ominus \pi_m^{-1}(2^m) \bigr) \oplus j_3 && \text{by } \eqref{oplusDefB} \\
    & = \pi_m\bigl ((\pi_m^{-1}(j_1) \oplus \pi_m^{-1}(j_2)  \ominus \pi_m^{-1}(2^m)) \\
    &\qquad \qquad \qquad \qquad \oplus \pi_m^{-1}(j_3) \bigr) && \text{by } \eqref{oplusDefA} \\
    & = \pi_m\bigl (\pi_m^{-1}(j_1) \\
    & \qquad  \oplus (\pi_m^{-1}(j_2) \oplus \pi_m^{-1}(j_3)\ominus \pi_m^{-1}(2^m)) \bigr) && \text{by the associative property on } \ci_m \\
    & = j_1 \oplus \bigl (\pi_m^{-1}(j_2) \oplus \pi_m^{-1}(j_3)) \ominus \pi_m^{-1}(2^m) \bigr) && \text{by } \eqref{oplusDefA} \\
    & = j_1 \oplus (j_2 \oplus j_3) && \text{by } \eqref{oplusDefB}.
\end{align*}
Thus, the associative property holds.

Under definition \eqref{oplusDefB} it follows that 
\begin{equation*}
    j = \pi_m(j \oplus 2^m),  \qquad \forall j \in \cj_m.
\end{equation*}
Thus, if $\oplus$ is defined first, then $\pi_m$ can be inferred from the equation above.



\section{Examples}

This first example corresponds to $i \oplus j$ defined by digit-wise addition:
\[
\begin{array}{ccccc|cccccccc}
&&&& \multicolumn{9}{c}{i \oplus j} \\
\pi_2(i) & \pi_1(i) &  \pi_0(i) & \phi(i) & \multicolumn{1}{c}{i\backslash j} & 0 & 1 & 2 & 3 & 4 & 5 & 6 & 7 \\ 
    \toprule
    4 & 2 & 1 & 0 & 0 & 0\\
    5 & 3 & & 0.5 & 1 & 1 & 0 \\
    6 & & & 0.25  & 2 & 2 & 3 & 0\\
    7 & & & 0.75  & 3 & 3 & 2 & 1 & 0\\
      & & & 0.125 & 4 & 4 & 5 & 6 & 7 & 0\\
      & & & 0.625 & 5 & 5 & 4 & 7 & 6 & 1 & 0\\
      & & & 0.375 & 6 & 6 & 7 & 4 & 5 & 2 & 3 & 0\\
      & & & 0.875 & 7 & 7 & 6 & 5 & 4 & 3 & 2 & 1 & 0\\
\end{array}
\]
A second example corresponds to $\phi(i \oplus j) = \phi(i) + \phi(j) \bmod 1$:
\[
\begin{array}{ccccc|ccccccccc}
&&&& \multicolumn{9}{c}{i \oplus j} \\
\pi_2(i) & \pi_1(i) &  \pi_0(i) & \phi(i) & \multicolumn{1}{c}{i\backslash j} & 0 & 1 & 2 & 3 & 4 & 5 & 6 & 7 \\ 
    \toprule
    7 & 3 & 1 & 0 & 0& 0\\
    6 & 2 & & 0.5 & 1 & 1 & 0 \\
    4 & & & 0.25  & 2 & 2 & 3 & 1\\
    5 & & & 0.75  & 3 & 3 & 2 & 0 & 1\\
      & & & 0.125 & 4 & 4 & 5 & 6 & 7 & 2\\
      & & & 0.625 & 5 & 5 & 4 & 7 & 6 & 3 & 2\\
      & & & 0.375 & 6 & 6 & 7 & 5 & 4 & 1 & 0 & 3\\
      & & & 0.875 & 7 & 7 & 6 & 4 & 5 & 0 & 1 & 2 & 3\\
\end{array}
\]

\end{document}