\documentclass[11pt,a4paper]{article}
\usepackage{latexsym,amsfonts,amsmath,graphics}
\usepackage{epsfig}
\usepackage[notref,notcite]{showkeys}
%\usepackage[active]{srcltx}
\usepackage{enumitem}
\usepackage[usenames,dvipsnames,svgnames,table]{xcolor}

\setlength{\textheight}{24cm} \setlength{\textwidth}{16cm}
\setlength{\hoffset}{-1.3cm} \setlength{\voffset}{-1.8cm}
\newtheorem{theorem}{Theorem}
\newtheorem{lemma}{Lemma}
\newtheorem{corollary}{Corollary}
\newtheorem{proposition}{Proposition}
\newtheorem{conjecture}{Conjecture}
\newtheorem{definition}{Definition}
\newtheorem{algorithm}{Algorithm}
\newtheorem{remark}{Remark}
\newenvironment{proof}{\begin{trivlist}
\item[\hskip\labelsep{\it Proof.}]}{$\hfill\Box$\end{trivlist}}

\newcommand{\norm}[1]{\left\Vert#1\right\Vert}
\newcommand{\To}{\rightarrow}
\newcommand{\Ref}[1] {\textup{(\ref{#1})}}
\newcommand{\satop}[2]{\stackrel{\scriptstyle{#1}}{\scriptstyle{#2}}}
\newcommand{\sump}{\mathop{{\sum}'}}
\newcommand{\bsDelta}{\boldsymbol{\Delta}}
\newcommand{\bsa}{\boldsymbol{a}}
\newcommand{\bszeta}{\boldsymbol{\zeta}}
\newcommand{\bsgamma}{\boldsymbol{\gamma}}
\newcommand{\bseta}{\boldsymbol{\eta}}
\newcommand{\bsalpha}{\boldsymbol{\alpha}}
\newcommand{\bslambda}{\boldsymbol{\lambda}}
\newcommand{\bsnu}{\boldsymbol{\nu}}
\newcommand{\bsmu}{\boldsymbol{\mu}}
\newcommand{\bsk}{\boldsymbol{k}}
\newcommand{\bsl}{\boldsymbol{l}}
\newcommand{\bsv}{\boldsymbol{v}}
\newcommand{\bsw}{\boldsymbol{w}}
\newcommand{\bsb}{\boldsymbol{b}}
\newcommand{\bsx}{\boldsymbol{x}}
\newcommand{\bsh}{\boldsymbol{h}}
\newcommand{\bsg}{\boldsymbol{g}}
\newcommand{\bst}{\boldsymbol{t}}
\newcommand{\bsz}{\boldsymbol{z}}
\newcommand{\bsu}{\boldsymbol{u}}
\newcommand{\bsq}{\boldsymbol{q}}
\newcommand{\bsone}{\boldsymbol{1}}
\newcommand{\tru}{{\rm tr}}
\newcommand{\maxp}{\operatornamewithlimits{max\phantom{p}}}
\newcommand{\infp}{\operatornamewithlimits{inf\phantom{p}}}
\newcommand{\bsy}{\boldsymbol{y}}
\newcommand{\bssigma}{\boldsymbol{\sigma}}
\newcommand{\D}{{\cal D}}
\newcommand{\E}{{\cal E}}
\newcommand{\cP}{{\cal P}}
\newcommand{\cG}{{\cal G}}
\newcommand{\cQ}{{\cal Q}}
\newcommand{\e}{{\varepsilon}}
\newcommand{\wal}{{\rm wal}}
\newcommand{\sob}{{\rm sob}}
\newcommand{\dsh}{{\rm dsh}}
\newcommand\setu{{\mathfrak{u}}}
\newcommand{\uu}{\mathfrak{u}}
\newcommand{\de}{{\rm e}}
\newcommand{\swal}{{\rm wal}}
\newcommand{\landau}{{\cal O}}
\newcommand{\mc}{{\rm mc}}
\newcommand{\icomp}{\mathtt{i}}
\newcommand{\bszero}{\boldsymbol{0}}
\newcommand{\rd}{\,\mathrm{d}}
\newcommand{\NN}{\mathbb{N}}
\newcommand{\ZZ}{\mathbb{Z}}
\newcommand{\integer}{\ZZ}
\newcommand{\RR}{\mathbb{R}}
\newcommand{\CC}{\mathbb{C}}
\newcommand{\FF}{\mathbb{F}}
\newcommand{\PP}{\mathbb{P}}
\newcommand{\QQ}{\mathbb{Q}}
\newcommand{\GG}{\mathbb{G}}
\newcommand{\calH}{\mathcal{H}}
\newcommand{\LL}{\mathcal{L}_{\bsg}}
\newcommand{\LLu}{\mathcal{L}_{\setu,\bsg}}
\newcommand{\qed} {\hfill \Box \vspace{0.5cm}}
\renewcommand{\pmod}[1]{\,(\bmod\,#1)}
\newcommand{\ns}{\negthickspace\negthickspace}
\newcommand{\nns}{\negthickspace\negthickspace\negthickspace\negthickspace}
\newcommand{\MC}{{\rm MC}_{n,s}}
\newcommand{\EE}{\mathbb{E}}
\newcommand{\il}{\left<}
\newcommand{\ir}{\right>}
\def\qed{\hfill$\Box$}
\newcommand{\abs}[1]{\left\vert#1\right\vert}
\def\calV{{\cal V}}


\DeclareMathOperator*{\esssup}{ess\,sup}

\allowdisplaybreaks

\newcommand{\TODO}[1]{{\color{red} Todo: #1}}


\begin{document}

\title{Lattices with net structure}

\maketitle

\section{General Idea}

Let $b$ be a prime, $m\in\NN$, and let $N=b^m$. Let $\FF_b$ be the finite field with $b$ elements. We associate 
the set $\{0,1,\ldots,b-1\}$ with the elements of $\FF_b$ in the obvious way.

For an integer $k\in\{0,1,\ldots,b^m-1\}$ with $b$-adic expansion
$$
k=\kappa_0 +\kappa_1 b + \cdots + \kappa_{m-1} b^{m-1}
$$
and an $m\times m$-matrix $A$ over $\FF_b$, define a mapping 
$$
 \nu_A (k):= \eta_0 + \eta_1 b + \cdots + \eta_{m-1} b^{m-1},
$$
where the $\eta_i$ are obtained by 
$$
A \cdot (\kappa_0,\kappa_1,\ldots,\kappa_{m-1})^\top=(\eta_0,\eta_1,\ldots,\eta_{m-1})^\top,
$$
with the matrix-vector product computed over $\FF_b$. 

\medskip

We obtain a point set of $N=b^m$ points in $[0,1)^d$ in the following way.

Let $C_1,\ldots,C_d$ be $d$ $m\times m$-matrices (the generating matrices) 
over $\FF_b$. Furthermore, let $\bsg=(g_1,\ldots,g_d)\in\ZZ^d$ be a generating vector. We then define 
points $\bsx_0,\bsx_1,\ldots,\bsx_{N-1}$ by
$$
 \bsx_k =\left(\left\{\frac{\nu_{C_1} (k) g_1}{N}\right\},\ldots,\left\{\frac{\nu_{C_d} (k) g_d}{N}\right\}\right)
$$
for $0\le k\le N-1$. Here, the terms 
$$
\frac{\nu_{C_j} (k) g_j}{N}
$$
are computed using normal integer arithmetic, and $\{\cdot\}$ denotes the fractional part. 


\medskip

\textbf{Special Cases:}

\begin{itemize}
\item If we choose $C_1,\ldots,C_d$ equal to the $m\times m$-identity matrix, we obtain a rank-1 lattice with $N$ points 
and generating vector $\bsg$. 
\item If we choose all $C_j$ equal to
$$
 \begin{pmatrix}
  0 &0 & \cdots & 0 & 1\\
  0 & 0&\cdots & 1 & 0\\
  \hdots\\
  0 & 1 & \cdots & 0 & 0 \\
  1 & 0 & \cdots & 0 & 0 
 \end{pmatrix},
$$
we obtain the first $N=b^m$ points of an extensible lattice point set in the sense of \cite{HH97}. 

\item If we choose $g_1=\cdots=g_d=1$, we obtain a digital net. 

\item If, for $1\le s< d$, we choose $g_1=\cdots=g_s=1$, $C_1,\ldots,C_s$ as the generating matrices of a digital net, 
$C_{s+1},\ldots, C_d$ equal to the identity matrix, and $g_{s+1},\ldots,g_d$ like the components of a generating vector 
of a $(d-s)$-dimensional rank-1 lattice point set, we obtain a hybrid point set in the sense of \cite{K12}. 
\end{itemize}


\medskip

\textbf{Questions:}

\begin{itemize}
 \item Do the point sets in their general form have a group structure?
 \item What is their quality with respect to common figures of merit?
 \item Can we use this concept to obtain a form of ``higher order'' lattice rules (analogous to Dick's higher order polynomial lattice rules)? 
 \item Can we make the definition above extensible in $N$ by using the generating matrices of digital sequences? 
 \item \ldots.
\end{itemize}


\begin{thebibliography}{99}
 \bibitem{HH97} F.J.~Hickernell, H.S.~Hong. Computing multivariate normal probabilities using rank-1 lattice sequences. 
 In: \textit{Scientific Computing (Hong Kong, 1997)}, 209--215. Springer, Singapore, 1997.
 
 \bibitem{K12} P.~Kritzer. On an example of finite mixed quasi-Monte Carlo point sets. Monatsh. Math. 168, 443--459, 2012. 
\end{thebibliography}


\end{document}